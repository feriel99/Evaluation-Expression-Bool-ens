\documentclass{article}
\usepackage[utf8]{inputenc}
\usepackage[T1]{fontenc}
\usepackage[french]{babel}
\usepackage{minted}
\usepackage{float}
\usepackage{tikz}
\usetikzlibrary{automata,positioning}
\begin{document}
\title{Projet IN406 - Évaluation d’expressions booléennes}

\author{BOUSSOUAR Feriel et MOUNGAD Massilia}
\date{25 mai 2021}

\newpage

\maketitle
\section*{Question 1 :}
Lire une chaine de caractère contenant une expression arithmétique et la transformer en une liste de tokens. \\
Nous avons créé et  stocké un token en utilisant  des enums, des struct et des fonctions suivantes:

\begin{minted}{C}
typedef enum { CONSTANTE, OPERATEUR, PARENTHESE } typeT;

typedef enum
{
    FAUX = 0, VRAI = 1,
    NON, ET, OU, IMPLICATION, EQUIVALENCE,
    GAUCHE , DROITE
}   valeurT;

typedef struct token* liste_token;
struct token
{
    typeT type;
    valeurT val;
    liste_token suiv;
}; 
\end{minted}
\begin{itemize}
    \item liste\_token allouer\_liste\_token(typeT type, valeurT val): permet d'allouer la mémoire à un token dans la liste.
    \item void desallouer\_liste\_token(liste\_token liste): permet de libèrer la liste de tokens en mémoire.
    \item liste\_token string\_to\_token(const char  *string): permet de transformer une chaîne de caractères en liste de token, elle teste à chaque itération le  caractère string de la  chaine pour lui attribuer un type et une valeur par exemple s'il est égale à 0 le type sera CONSTANTE de valeur v = FAUX afin de le stocker dans un token.
\end{itemize}

\section*{Question 2 :}
Donner un automate à pile reconnaissant le langage dont les mots sont les expressions booléennes.
\vspace{5px}

Soit L le langage : \\
$L = \{\; w\;g\acute{e}n\acute{e}r\acute{e}e\;par\;G,\; |w|_{(} = |w|_{)}\;\}$\\
le nombre de parenthèses ouvrantes = le nombre de parenthèses fermantes 
\vspace{5px}\\*

Définition formelle : \\
\vspace{5px}\\*
\begin{math}
    A = (\Sigma,\;Q,\;\Gamma,\;q_0,	\delta,\;F,\;T)\\
	\Sigma =\{\;0,\;1,\;NON,\;+,\;.,\;\Rightarrow,\;\Leftrightarrow\;\}\\
	Q = \{\;q_0,\;q_1\;\}\\
    \Gamma = \{\;X\;\}\\
    \delta \\
    q0 = \{\;q0\;\}\\
    F = \{\;q_1\;\}\\
    T = \{ \\
    \hspace*{7ex} (q_0,\;\delta,\;NON,\;\delta,\;q_0),\;(q_0,\;\delta,\;(,X,\;q_0\;), \\
    \hspace*{7ex} (q_0,\;\delta,\;0,\;\delta,\;q_1\;),\;(q_0,\;\delta,\;1,\;\delta,\;q_1\;), \\
    \hspace*{7ex} (q_1,\;\delta,\;+,\;\delta,\;q_0),\;(q_1,\;\delta,\;.,\;\delta,\;q_0), \\
    \hspace*{7ex} (q_1,\;\delta,\;\Rightarrow,\;\delta,\;q_0),\;(q_1,\;\delta,\;\Leftrightarrow,\;\delta,\;q_0), \\
    \hspace*{7ex} (q_1,\;X,\;),\;\epsilon,\;q_1), \\
    \}
\end{math}

\begin{figure}[H]
\begin{tikzpicture}
   \node[state,initial] (q_0)  {$q_0$}; 
   \node[state, accepting] (q_1) [right=4cm] {$q_1$}; 
    \path[->] 
    (q_0)   edge [loop below] node[below=0.0] {\delta,$NON,\delta$}
                              node[below=0.7] {$(,\delta,X$} ()
            edge [bend left=30] node[above=1] {X,$0,X$}
                                node[above=0.5] {X,$1,X$} (q_1)
    (q_1)   edge [loop above] node {X,$),\epsilon$} ()
            edge [bend left=30] node[below=0.5] {X,$+,X$}
                                node[below=1] {X,$.,X$}
                                node[below=1.5] {X,$\Rightarrow,X$}
                                node[below=2] {X,$\Leftrightarrow,X$} (q_0);
\end{tikzpicture}
\caption{Automate à pile reconnaissant le langage par état final et pile vide.}
\end{figure} 
\begin{itemize}
    \item Si on lit une parenthese ouvrante  ( : on empile dans la pile  et on reste dans q0
    \item Si on lit NON : on fait rien dans la pile et on reste dans  q0
    \item Si on lit 0 ou 1 : on fait rien dans la pile et on va de q0 à q1
    \item Si on lit une parenthese fermante ) : on depile et on reste dans q1
    \item Si on lit + ou . ou  \begin{math} \Rightarrow \end{math}  ou \begin{math} \Leftrightarrow{} \end{math} :  on fait rien dans la pile et on va  de q1 à q0 
\end{itemize}

\section*{Question 3 :}

La fonction qui teste si une liste de token appartient au langage ou non\\
int expression\_appartient\_langage(liste\_token liste): elle prend en paramètre une liste et elle vérifie si l'expression est valide 


\section*{Question 4:}

Créer l'arbre représentant l'expression booléenne\\
La strucutre pour stocker un arbre des tokens: 
\begin{minted}{C}
typedef struct arbre* arbre_token;
struct arbre
{
    typeT type;
    valeurT val;
    arbre_token gauche;
    arbre_token droite;
};
\end{minted}
Struct arbre a deux  pointeurs de type struct arbre* pour le fils gauche 
et le fils droit.\\
Les fonction pour créer l'arbre qui représente l'expréssion booléene: \\
\begin{itemize}
    \item arbre\_token alloue\_arbre\_token(liste\_token liste): permet d'allouer la memoire à un token pour le stocker dans l'arbre
    \item arbre\_token liste\_token\_to\_arbre\_token(liste\_token l): permet de transformer une liste de tokens en arbre 
\end{itemize}

\newpage
\section*{Question 5 :}
\vspace{5px}

Calculer la valeur de l expression arithmétique et afficher le résultat \\
\begin{itemize}
 \item int calculer(valeurT x, valeurT y, valeurT op), nous avons fait dans cette 
fonction un switch pour calculer et retourner une valeur  de a et b en fonction de 
type de l opérateur.
\end{itemize}
\begin{minted}{C}
 int calculer(valeurT x, valeurT y, valeurT op)
{   
	switch (op)
	{ 
	    case (NON) :
	        return ! x;
	    case (OU) :
	        return x|y; 
	    case (ET) :
	        return x&y; 
	    case (IMPLICATION) :
	        return (! x) | y;
	    case (EQUIVALENCE) :
	        return ! x ^ y;
	}
} 
\end{minted}

\begin{itemize}
    \item int arbre\_to\_int(arbre\_token arb): permet de calculer la valeur de l'expression booléenne stockée dans la liste des tokens dans l arbre en faisant un appel récursif de la fonction calculer pour calculer la valeur entre fils gauche et fils droit.
\end{itemize}
 
\begin{minted}{C}
int arbre_to_int(arbre_token arb)
{
    if (! arb)   return 0;
    else if (arb->type == CONSTANTE)
        return at->valeur;
    return calcule(arbre_to_int(at->gauche), arbre_to_int(arb->droite), arb->valeur);
}
\end{minted} 

\newpage
\section*{Commentaires Supplémentaires :}
En plus de la fonction liste\_token liste\_token\_to\_arbre(liste\_token liste), nous avons
développé la fonction liste\_token liste\_token\_postfixe(liste\_token liste) permettant de
transformer la liste token en liste postfixe à l'aide d'une pile en utilisant les macros PUSH et POP pour empiler et dépiler dans la pile afin de pouvoir évaluer notre expression booléene.

\end{document}